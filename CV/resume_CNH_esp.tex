\documentclass{resume} % Use the custom resume.cls style

\usepackage[left=0.4 in,top=0.4in,right=0.4 in,bottom=0.4in]{geometry} % Document margins
\usepackage{multicol}
\newcommand{\tab}[1]{\hspace{.2667\textwidth}\rlap{#1}} 
\newcommand{\itab}[1]{\hspace{0em}\rlap{#1}}
\name{César Núñez Huamán} % Your name
% You can merge both of these into a single line, if you do not have a website.
\address{+51 990 992 899 \\ +1 (773) 886-8445 \\ \href{mailto:cnunezh@uchicago.edu}{cnunezh@uchicago.edu} \\ \href{https://github.com/cesarnunezh}{GitHub} \\ \href{https://linkedin.com/in/cesar-nunez-huaman}{LinkedIn}}
%\address{(773) 886-8445 \\ \href{mailto:cnunezh@uchicago.edu}{cnunezh@uchicago.edu} \\ \href{https://github.com/cesarnunezh}{GitHub} \\ \href{https://linkedin.com/in/cesar-nunez-huaman}{LinkedIn} \\ \href{https://cesarnunezh.github.io/}{Web page}}

\begin{document}

%----------------------------------------------------------------------------------------
%	EDUCATION SECTION
%----------------------------------------------------------------------------------------

\begin{rSection}{Educación}

{\bf Harris School of Public Policy, The University of Chicago} \hfill \textit{Chicago, IL} \\
\textit{Master of Science in Computational Analysis and Public Policy} \hfill {Junio 2026}

{\bf Universidad del Pacífico} \hfill \textit{Lima, Perú} \\
\textit{Licenciado en Economía} \hfill {Marzo 2020} \\
Tesis: El efecto de la capacitación docente sobre las prácticas pedagógicas y los aprendizajes. \href{https://repositorio.up.edu.pe/handle/11354/2653}{Disponible aquí.}

{\bf Universidad del Pacífico} \hfill \textit{Lima, Perú} \\
\textit{Bachiller en Economía} \hfill {Marzo 2018} \\
Tercio superior. Concentración en economía para el sector público.

\end{rSection}

%----------------------------------------------------------------------------------------
%	WORK EXPERIENCE SECTION
%----------------------------------------------------------------------------------------
\begin{rSection}{Experiencia profesional}

    \textbf{Videnza Consultores} \hfill Lima, Perú \\
    \textit{Analista senior} \hfill \textit{Jul 2020 - Ago 2024}
     \begin{itemize}
        \itemsep -3pt {} 
         \item Lideró el equipo que brindó asistencia técnica en el diseño de la Estrategia de Reducción de la Pobreza Urbana para el Ministerio de Desarrollo e Inclusión Social del Perú mediante el desarrollo de una teoría de cambio y un modelo conceptual, utilizando revisiones sistemáticas y de literatura de más de 50 artículos y análisis de componentes principales con datos de encuestas de hogares (con +500k observaciones) para asegurar intervenciones efectivas basadas en evidencia para la reducción de la pobreza urbana. \\ \textbf{Herramientas utilizadas:} Python, Pandas, Geopandas, R, Tidyverse, Principal Component Analysis. 
         \item Formó parte de la implementación del Observatorio Bicentenario, un repositorio de estadísticas nacionales y regionales con más de 150 indicadores, utilizando datos censales, de encuestas y administrativos para diseñar indicadores y desarrollar visualizaciones en Python, R y STATA, mejorando la accesibilidad de los datos para informar las decisiones políticas y promover la gobernanza basada en evidencia. \href{https://propuestasdelbicentenario.pe/observatorio/}{Disponible aquí.} \\ \textbf{Herramientas utilizadas:} Python, Pandas, Selenium, R, STATA.
         \item Gestión de la recopilación de datos y la implementación de un ensayo controlado aleatorio para mejorar la adquisición de suministros médicos en 140 centros de salud públicos. Diseño y evaluación del impacto de una intervención basada en la ciencia del comportamiento, lo que dio lugar a una mayor tasa de adjudicación de compras y a una reducción de los costos de los suministros. \\ \textbf{Herramientas utilizadas:} STATA.
     \end{itemize}
     
     \textbf{Consejo Privado de Competitividad (CPC)} \hfill Lima, Perú\\
     \textit{Analista} \hfill \textit{Jul 2019 - Jun 2020}
     \begin{itemize}
        \itemsep -3pt {} 
         \item Lideré las actualizaciones anuales y trimestrales del Índice de Gestión Pública Regional (IGPR), que mide el desempeño de la gestión pública regional y local en las 25 regiones del Perú. Utilicé datos administrativos y de encuestas para calcular 17 indicadores clave de desempeño, cuyos resultados se utilizaron para impulsar mejoras en las prácticas de gestión de las autoridades regionales. Disponible \href{https://www.compite.pe/wp-content/uploads/2020/07/IRGP-2020-version-final.pdf}{aquí} \\ \textbf{Herramientas utilizadas:} STATA.
         \item Desarrollo de investigación y propuestas de políticas públicas basadas en evidencia para fomentar sinergias público-privadas. Las iniciativas clave incluyeron la reforma de las regulaciones para el desarrollo de infraestructura mediante la agilización de los procesos administrativos para acelerar los proyectos de gran escala en el marco del Plan Nacional de Infraestructura del Perú, y el diseño de un plan de estímulo económico para ampliar los subsidios de vivienda durante la pandemia, mejorando el acceso a la vivienda asequible e impulsando la recuperación económica.
     \end{itemize}
    \end{rSection} 
     \pagebreak

\begin{rSection}{Experiencia profesional}
    \textbf{AC Pública, subsidiaria de APOYO Consultoría} \hfill Lima, Perú\\
    \textit{Consultor} \hfill \textit{Ene 2018 - Jul 2019}
     \begin{itemize}
        \itemsep -3pt {} 
         \item Contribuyó al desarrollo y aprobación de la Política y Plan Nacional de Innovación Agraria mediante el análisis de datos de encuestas y censos, la realización de investigaciones bibliográficas y económicas y la realización de trabajo cualitativo de campo para informar las decisiones políticas y garantizar la alineación con los objetivos nacionales de desarrollo.
         \item Dirigió la revisión de las estimaciones de oferta y demanda para los Planes Maestros de Desarrollo de 12 aeropuertos regionales, utilizando modelos econométricos para mejorar la precisión, garantizar el cumplimiento de las normas regulatorias e impulsar la planificación estratégica para el crecimiento sostenible de la infraestructura.
     \end{itemize}

     \textit{Practicante} \hfill \textit{Abr - Dic 2017}
     \begin{itemize}
        \itemsep -3pt {} 
         \item Apoyo en el desarrollo de los proyectos de consultoría, incluyendo trabajo de oficina y trabajo de campo, incluyendo el manejo de base de datos, elaboración de instrumentos de recolección de información y sistematización de datos cuantitativos y cualitativos.
         \item Apoyo en la gestión de proyectos ejecutados mediante el mecanismo de Obras por Impuestos.
     \end{itemize}

    \textbf{Gabinete de Asesores del Despacho Ministerial - Ministerio de Economía y Finanzas} \hfill Lima, Perú\\
    \textit{Practicante} \hfill \textit{Dic 2016 - Mar 2017}
    \begin{itemize}
        \itemsep -3pt {} 
        \item Apoyo en la elaboración del diagnóstico del Sistema de Protección Social del Perú, como soporte de la Comisión de Protección Social.
        \item Brindar seguimiento a los proyectos de Asociaciones Público Privada, a través de presentaciones y tableros de control.
        \item Brindar seguimiento a los temas relacionados al MEF que se traten en las comisiones del Congreso de la República (Decretos Legislativos y Proyectos de Ley).
    \end{itemize}

    \textbf{Instituto Peruano de Economía (IPE)} \hfill Lima, Perú\\
    \textit{Practicante} \hfill \textit{May 2016 - Nov 2016}
    \begin{itemize}
        \itemsep -3pt {} 
        \item Elaboración de reportes mensuales a los asociados y de coyuntura económica regional. 
        \item Participación en la elaboración del Índice Compuesto de Actividad Económica Regional (ICAE) y el Índice de Competitividad Regional (INCORE). 
        \item Apoyo en la elaboración de estudios y consultorías encargadas al IPE.
    \end{itemize}
\end{rSection} 

%----------------------------------------------------------------------------------------
% SKILLS
%----------------------------------------------------------------------------------------
\begin{rSection}{Habilidades técnicas}
    \begin{tabular}{@{} l l @{}}
    \textbf{Lenguajes de programación:} & Python, R, HTML, \LaTeX, SQL\\
    \textbf{Análisis estadístico:} & STATA, R \\
    \textbf{Idiomas:} & Spanish (Native), English (Advanced), French (Basic)
    \end{tabular}
    \end{rSection}

%----------------------------------------------------------------------------------------
%	CURSOS Y PROGRAMAS DE ESPECIALIZACIÓN
%----------------------------------------------------------------------------------------

\begin{rSection}{Cursos y Programas de Especialización}

{\bf Massachusetts Institute of Technology - MIT} \hfill {2025} \\
MicroMasters Program on Data, Economics and Development Policy

{\bf Pontificia Universidad Católica del Perú - PUCP} \hfill {2023}\\
Diplomatura de Especialización en Ciencia de Datos para las Ciencias Sociales y la Gestión Pública

{\bf Pontificia Universidad Católica del Perú - PUCP} \hfill {2021}\\
Programa de Gobernabilidad, Gerencia Política y Gestión Pública

{\bf Grupo de Análisis para el Desarrollo - GRADE} \hfill {2019}\\
Evaluación de Programas Sociales

\end{rSection}

\pagebreak

%----------------------------------------------------------------------------------------
%	TEACHING EXPERIENCE
%----------------------------------------------------------------------------------------

\begin{rSection}{Experiencia docente}

{\bf Universidad del Pacífico}  \hfill {2016 - 2023}\\
Jefe de prácticas en los siguientes cursos: Economía General I (5 semestres), Microeconomía I (1 semestre), Economía General II (9 semestres).
    
\end{rSection}

%----------------------------------------------------------------------------------------
% PROYECTOS Y PUBLICACIONES	
%----------------------------------------------------------------------------------------

% \begin{rSection}{Proyectos y consultorías}
% \vspace{-1.25em}
% \item \textbf{Política y Plan Nacional de Innovación Agraria (Ene 2018 - Ene 2019).} {Consultoría en el marco del Programa Nacional de Innovación Agropecuaria cuyo propósito fue desarrollar un diagnóstico situacional de la innovación agropecuaria en el Perú y, con base en dicho diagnóstico, formular la Política y Plan Nacional de Innovación Agropecuaria.}
% \item \textbf{Consultoría para el diseño de modelos operativos para la optimización de los procesos de gestión educativa en las Unidades de Gestión Educativa Local (Nov 2018 - Mar 2019).} {Consultoría para diseñar un nuevo modelo operativo para las Unidades de Gestión Educativa Local en Perú, con el fin de agilizar, simplificar y estandarizar cuatro procesos administrativos priorizados, que contribuyen a mejorar la calidad educativa.}
% \item \textbf{Actualización de los Planes Maestros de Desarrollo de 12 aeropuertos gestionados por Aeropuertos del Perú (Dic 2018 - Jul 2019).} {Consultoría para la actualización de las estimaciones de oferta y demanda de los Planes Maestros de Desarrollo (PMD) de los 12 aeropuertos concesionados por Aeropuertos del Perú. A partir de las proyecciones de demanda de pasajeros y carga, se determinó en qué año sería necesario ampliar los aeropuertos. Para obtener más información sobre los PMD, \href{https://www.adp.com.pe/es/nuestros-proyectos}{haga click aquí.}}
% \item \textbf{Consultoría para el desarrollo de estrategias que fortalezcan a la Universidad Católica Sedes Sapientiae en su filial Chulucanas (Mar 2019 - Jul 2019).} {Consultoría para desarrollar un estudio de oferta y demanda para la implementación de un nuevo campus universitario de la Universidad Católica Sedes Sapientiae (UCSS) en su sede en Chulucanas, Piura. Actualmente, la UCSS se encuentra en proceso de diseño y construcción de las nuevas instalaciones del nuevo campus en Chulucanas.}
% \item \textbf{Análisis de la situación de la niñez y adolescencia en Perú para UNICEF Perú (Jul 2020 - Feb 2021).} {Consultoría para desarrollar el análisis de la situación de la niñez y la adolescencia para el nuevo Programa de Cooperación entre el Fondo de las Naciones Unidas para la Infancia (UNICEF) y el Estado Peruano. Este análisis se centró en los derechos a la salud, la educación, la protección social y la protección contra la violencia. Puede acceder al resumen ejecutivo \href{https://www.unicef.org/peru/media/12141/file/Resumen Ejecutivo: Situaci%C3%B3n de ni%C3%B1as, ni%C3%B1os y adolescentes en el Per%C3%BA .pdf}{aquí.}}
% \item \textbf{Proyecto de investigación sobre el desempeño de los funcionarios públicos a cargo de los procesos generales de contratación en el Perú (Jul 2020 - Ene 2021).} {Consultoría con el objetivo de identificar estrategias encaminadas a lograr mejoras en los procesos de contratación pública de insumos médicos. En concreto, se diseñó una intervención basada en las ciencias del comportamiento y se diseñó un plan de evaluación de impacto para medir los resultados de dicha intervención.}
% \item \textbf{Propuesta de indicadores cruciales para el seguimiento de la gestión institucional del Seguro Social de Salud – EsSalud (Dic 2020 - Feb 2021).} {Consultoría para diseñar un tablero de indicadores de gestión que faciliten el seguimiento y análisis de las prestaciones en salud, para que la dirección ejecutiva de EsSalud pueda tomar mejores decisiones y mejorar la prestación de los servicios de salud.}
% \item \textbf{Consultoría para la Segunda Medición del Proyecto Integración de Niños, Niñas y Adolescentes Venezolanos en Perú para UNICEF Perú (Ene 2021 - Mar 2021).} {Consultoría para evaluar la situación final del proyecto “Integración de Niñas, Niños y Adolescentes Venezolanos en Perú” implementado por UNICEF Perú. La evaluación se realizó a partir de la medición de indicadores a partir de información primaria y secundaria; así como del desarrollo de tres casos de estudio.}
% \item \textbf{Consultoría para la medición de resultados del estudio de intervenciones de apoyo a procesos de compras públicas en Perú (Mar 2021 - Ene 2022).} {Consultoría para implementar y monitorear la intervención diseñada para mejorar los procesos de compras públicas de insumos médicos, así como para realizar una evaluación de impacto de la intervención. La intervención permitió un mayor índice de adjudicación de compras y una reducción en los precios unitarios de las compras de insumos médicos.}
% \end{rSection} 

% \pagebreak
% \begin{rSection}{Proyectos y consultorías}
% \item \textbf{Elaboración de un análisis de impacto de la modificación del crédito a la EPS contra los aportes a EsSalud (Jun 2021 - Jul 2021).} {Consultoría para realizar un análisis de sensibilidad de los ingresos de EsSalud ante posibles cambios en el nivel de crédito que reciben las Empresas Prestadoras de Salud. Asimismo, se realizó un análisis del impacto financiero de dichos cambios y se realizaron propuestas para asegurar la sostenibilidad financiera de la institución.}
% \item \textbf{Formulación del Observatorio de la iniciativa “Propuestas del Bicentenario” de Videnza Consultores para las temáticas de Estado Eficiente, Industrias Extractivas y Agricultura (Jun 2021 - Oct 2021).} {Proyecto interno de Videnza Consultores en el marco de la iniciativa “Propuestas Bicentenario”, con el propósito de generar un observatorio de indicadores que permita evaluar y dar seguimiento a 8 ejes clave con información a nivel nacional y regional. Para más información sobre el observatorio, \href{https://propuestasdelbicentenario.pe/observatorio/}{ingresa aquí.}}
% \item \textbf{Consultoría para la elaboración de un documento de discusión sobre la generación de valor en el entorno de los proyectos mineros en el Perú y sus implicaciones en el conflicto (Nov 2021 - Feb 2022).} {Consultoría para identificar las condiciones y características necesarias para la generación de valor en entornos donde se desarrollan actividades mineras; así como identificar la relación que existe entre estas características y el conflicto social.}
% \item \textbf{Consultoría para la elaboración de un manual de conceptos sobre relaciones sociales en el Corredor Sur Minero para MMG Las Bambas (Mar 2022 - May 2022).} {Consultoría para desarrollar un manual de conceptos que siente las bases para diseñar e implementar una estrategia de relacionamiento efectivo de la empresa MMG Las Bambas con las instituciones y actores del corredor vial sur (Apurímac - Cotabambas, Cusco - Chumbivilcas y Arequipa).}
% \item \textbf{Consultoría para la identificación del arreglo institucional para la implementación del Programa en el Corredor Minero Sur (May 2022 - Jul 2022).} {Consultoría para desarrollar un análisis de los posibles diseños u opciones institucionales para la implementación del programa de inversiones bajo un enfoque territorial, así como para realizar un análisis comparativo de los diseños u opciones institucionales identificados.}
% \item \textbf{Estimación del impacto fiscal y económico de la transición energética en el Perú (Ago 2022 - Dic 2022).} {Consultoría para desarrollar la estimación del impacto fiscal y económico de la transición energética de la matriz eléctrica peruana, en términos de pérdida de recaudación de regalías gasíferas.}
% \item \textbf{Consultoría para la medición de resultados del estudio RG-T1736: Intervención para apoyar los procesos de compras públicas de baja cuantía en Perú (Set 2022 - Dic 2022).} {Consultoría para implementar y monitorear la intervención diseñada para mejorar los procesos de compras públicas de menor cuantía en entidades públicas del sector salud, así como para realizar una evaluación de impacto de la intervención.}
% \item \textbf{Asistencia Técnica para la integración de los procesos que conforman la Programación Multianual de Inversiones y la Programación Presupuestaria Multianual en materia de inversiones (Set 2022 - Ene 2023).} {Consultoría para elaborar una propuesta de optimización de la automatización e interrelación entre los procesos de la Programación Multianual de Inversiones y la Programación Presupuestaria Multianual en materia de inversiones}
% \item \textbf{Asistencia Técnica para la exploración de las opciones de instrumentos financieros que contribuyen a la conservación y el uso sostenible de la biodiversidad, con especial atención en los insectos polinizadores (Dic 2022 - Feb 2023).} {Estudio exploratorio sobre las opciones de instrumentos financieros que contribuyen a la conservación y el uso sostenible de la biodiversidad, con especial atención en los insectos polinizadores en Perú, México, Costa Rica y Brasil.}
% \end{rSection} 

% \pagebreak
% \begin{rSection}{Proyectos y consultorías}
% \item \textbf{Implementación de la tercera medición del índice de capital social 2023 a organizaciones comunitarias del ámbito DAIS (Desarrollo Alternativo Integral y Sostenible) y análisis comparativo de resultados de las tres mediciones (Mar 2023 - Jul 2023).} {Consultoría para desarrollar la tercera medición del Índice de Capital Social a 24 organizaciones de productores y 34 organizaciones vecinales en el ámbito del Proyecto USAID de Fortalecimiento Institucional de DEVIDA; así como la sistematización del proceso de diseño y aplicación del Índice de Capital Social en 2019, 2021 y 2023. Este trabajo implicó la revisión documentaria y un trabajo de campo cualitativo para la identificación de fortalezas y oportunidades de mejora en la aplicación del ICS.}
% \item \textbf{Consultoría para la sistematización de experiencias o prácticas de gestión pública del Proyecto de USAID: Fortalecimiento Institucional de DEVIDA (FID) (Abr 2023 - Sep 2023).} {Consultoría para la sistematización de cinco experiencias de gestión pública del Proyecto FID. Las experiencias priorizadas por el Proyecto FID fueron: i) fortalecimiento de la gestión de contrataciones, ii) fortalecimiento de organizaciones de productores mediante la herramienta VEO, iii) promoción de la obtención de financiamiento en organizaciones comunitarias, iv) fortalecimiento de la gestión de inversiones en municipalidades y v) obtención de recursos adicionales municipales para la inversión pública.  Este trabajo implicó la revisión documentaria y un trabajo de campo cualitativo para la identificación de fortalezas y oportunidades de mejora en las cinco experiencias priorizadas por el Proyecto FID. }
% \item \textbf{Asistencia técnica para el diseño de un documento técnico marco para la reducción de la pobreza urbana (Dic 2023 - Jul 2024)} {Brindar asistencia técnica para el diseño de la Estrategia de Reducción de la Pobreza Urbana por medio de la formulación de una teoría de cambio y modelo conceptual para la reducción de la pobreza urbana, basada en evidencia de intervenciones efectivas y eficientes.}
% \end{rSection} 
%----------------------------------------------------------------------------------------
% \pagebreak
% \begin{rSection}{Publicaciones}
% Durante mi experiencia profesional he colaborado con las siguientes publicaciones:
% \item Bustamante, P. (2022) \emph{La necesaria viabilidad social para el desarrollo sostenible de la minería}. En L. M. Castilla, J. Seinfield, N. Besich, C. Trivelli, J. Gallardo, A. Matsuda, . . . R. Valencia, \textit{Propuestas del Bicentenario: Rutas para el desarrollo regional} (págs. 215-241). Lima: Penguin Random House. Disponible \href{https://www.casadellibro.com/ebook-propuestas-del-bicentenario-ebook/9786125068026/13125397}{aquí.} 
% \item Castilla, L. M. (2021) \emph{Reactivación Económica para el crecimiento sostenido}. En L. M. Castilla, J. Seinfield, M. von Hese, N. Besich, M. Jaramillo, R. Barrantes, . . . D. Alfaro, \textit{Propuestas del Bicentenario: Rutas para un país en desarrollo} (págs. 29-64). Lima: Penguin Random House. Disponible \href{https://www.casadellibro.com/ebook-propuestas-del-bicentenario-ebook/9786124272813/12336851}{aquí.}
% \item Castilla, L. M. (2022) \emph{Encadenamientos productivos y desarrollo regional: el caso de la minería.}.  En L. M. Castilla, J. Seinfield, N. Besich, C. Trivelli, J. Gallardo, A. Matsuda, . . . R. Valencia, \textit{Propuestas del Bicentenario: Rutas para el desarrollo regional} (págs. 215-241). Lima: Penguin Random House. Disponible \href{https://www.casadellibro.com/ebook-propuestas-del-bicentenario-ebook/9786125068026/13125397}{aquí.} 
% \item Castilla, L. M., Seinfeld, J., von Hese, M., Besich, N. (2021) \emph{Introducción}. En L. M. Castilla, J. Seinfield, M. von Hese, N. Besich, M. Jaramillo, R. Barrantes, . . . D. Alfaro, \textit{Propuestas del Bicentenario: Rutas para un país en desarrollo} (págs. 29-64). Lima: Penguin Random House. Disponible \href{https://www.casadellibro.com/ebook-propuestas-del-bicentenario-ebook/9786124272813/12336851}{aquí.}
% \item Consejo Privado de Competitividad. (2019). \emph{Informe de Competitividad 2020}. Lima: Consejo Privado de Competitividad. Disponible \href{https://www.compite.pe/wp-content/uploads/2019/11/CPC_Peru_INC-2020_Libro-Web-Paginas.pdf}{aquí.}
% \item Consejo Privado de Competitividad. (2020). \emph{Índice Regional de Gestión Pública (IRGP) 2020}. Lima: Consejo Privado de Competitividad. Disponible \href{https://www.compite.pe/wp-content/uploads/2020/07/IRGP-2020-version-final.pdf}{aquí.}
% \item Consejo Privado de Competitividad. (2020). \emph{Informe de Competitividad 2021}. Lima: Consejo Privado de Competitividad. Disponible \href{https://www.compite.pe/wp-content/uploads/2021/01/Informe-de-Competitividad-2021-CPC.pdf}{aquí.}
% \end{rSection} 

\begin{rSection}{Actividades extra curriculares} 
\begin{itemize}
    \item \textbf{Latin American Matters (LAM) 2024}. LAM es una una organización estudiantil que busca promover la discusión y mejor comprensión de las cuestiones económicas, sociales y políticas de los países de América Latina a través de un enfoque de políticas públicas. Miembro de la Comisión de Desarrollo Profesional.
    \item \textbf{Beca Generación Bicentenario 2024}. Otorgado por el Gobierno peruano como parte del programa nacional de becas que reconocen la excelencia académica. Primer puesto del concurso del año 2024.
    \item 	\textbf{EvalYouth Global Mentoring Program (EY-GMP) 2020-2021.} {El EY-GMP es una iniciativa para apoyar a los evaluadores novatos y jóvenes profesionales para que se conviertan en profesionales capacitados y confiables que puedan asumir roles de evaluación con confianza en sus comunidades y países.}
    \item 	\textbf{Impacta! Jóvenes por la gestión pública 2019-2020.} {Impacta es una asociación civil sin fines de lucro, dirigida e integrada por jóvenes convencidos de que para lograr el pleno desarrollo de un país es necesario contar con instituciones públicas sólidas, transparentes y capaces de responder con eficiencia a las necesidades de sus ciudadanos.}
  \end{itemize}


\end{rSection}


   \end{document}
