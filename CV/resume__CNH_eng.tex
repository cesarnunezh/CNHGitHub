\documentclass{resume} % Use the custom resume.cls style

\usepackage[left=0.4 in,top=0.4in,right=0.4 in,bottom=0.4in]{geometry} % Document margins
\newcommand{\tab}[1]{\hspace{.2667\textwidth}\rlap{#1}} 
\newcommand{\itab}[1]{\hspace{0em}\rlap{#1}}
\name{César Núñez Huamán} % Your name
% You can merge both of these into a single line, if you do not have a website.
\address{+51 990 992 899 \\ Lima, Perú} 
\address{\href{mailto:ca.nunezhuaman@gmail.com}{ca.nunezhuaman@gmail.com} \\ \href{https://linkedin.com/in/cesar-nunez-huaman}{linkedin.com/in/cesar-nunez-huaman}}  %



\begin{document}

\item {Licenciado en Economía con concentración en economía para el sector público y con 5 años de experiencia en consultoría en políticas públicas, políticas sociales y economía. Con estudios en métodos cualitativos y cuantitativos aplicados a la evaluación de programas sociales y en gestión pública. Asimismo, cuento con experiencia en el manejo de bases de datos en STATA, R y Python, aplicado a las ciencias sociales.}

%----------------------------------------------------------------------------------------
%	EDUCATION SECTION
%----------------------------------------------------------------------------------------

\begin{rSection}{Educación}

{\bf Licenciado en Economía}, Universidad del Pacífico - UP \hfill {2020} \\
Tesis: El efecto de la capacitación docente sobre las prácticas pedagógicas y los aprendizajes. \href{https://repositorio.up.edu.pe/handle/11354/2653}{Disponible aquí.}

{\bf Bachiller en Economía}, Universidad del Pacífico - UP \hfill {2013 - 2017}\\
Tercio superior. Concentración en economía para el sector público.

\end{rSection}

%----------------------------------------------------------------------------------------
%	CURSOS Y PROGRAMAS DE ESPECIALIZACIÓN
%----------------------------------------------------------------------------------------

\begin{rSection}{Cursos y Programas de Especialización}

{\bf MicroMasters Program on Data, Economics and Development Policy}, Massachusetts Institute of Technology - MIT \hfill {2022 - 2023}\\
Actualmente cursando el programa, habiendo aprobado satisfactoriamente 3/5 cursos requeridos.

{\bf Diplomatura de Especialización en Ciencia de Datos para las Ciencias Sociales y la Gestión Pública}, Pontificia Universidad Católica del Perú - PUCP \hfill {2022 - 2023}\\
Actualmente cursando el programa.

{\bf Programa de Gobernabilidad, Gerencia Política y Gestión Pública}, Pontificia Universidad Católica del Perú - PUCP \hfill {2020 - 2021}\\
Culminado con aprobación satisfactoria.

{\bf Evaluación de Programas Sociales}, Grupo de Análisis para el Desarrollo - GRADE \hfill {2019}\\
Culminado con el certificado de "Aprobación sobresaliente"

\end{rSection}

%----------------------------------------------------------------------------------------
%	WORK EXPERIENCE SECTION
%----------------------------------------------------------------------------------------
\begin{rSection}{Experiencia profesional}

\textbf{Analista Senior} \hfill Jul 2020 - Actualidad\\
Videnza Consultores \hfill \textit{Lima, Perú}
 \begin{itemize}
    \itemsep -3pt {} 
     \item Coordinar, diseñar, planificar y liderar la ejecución de más de 15 proyectos de consultoría, que incluyen el análisis y sistematización de datos primarios y secundarios.
     \item Liderar el diseño e implementación de tres ejes temáticos en el “Observatorio del Bicentenario”, un repositorio de estadísticas nacionales y regionales lanzado en octubre de 2021 con más de 150 indicadores. \href{https://propuestasdelbicentenario.pe/observatorio/}{Disponible aquí.}
    \item Colaborar exitosamente con la publicación de dos libros bajo la iniciativa “Propuestas del Bicentenario” que presenta propuestas de política pública para enfrentar los principales desafíos del Perú para contribuir al desarrollo económico, social y ambiental. 
 \end{itemize}
 
 \textbf{Analista} \hfill Jul 2019 - Jun 2020\\
Consejo Privado de Competitividad (CPC) \hfill \textit{Lima, Perú}
 \begin{itemize}
    \itemsep -3pt {} 
     \item Liderar la actualización anual y trimestral del Índice de Gestión Pública Regional (IRGP) que mide el desempeño de la gestión pública regional y local en las 25 regiones del Perú y fue publicado en Diario Gestión en varias ocasiones.
     \item Dar seguimiento a las propuestas de política pública de la CPC, las cuales más del 69\% se encuentran dentro del Plan Nacional de Competitividad y Productividad, leyes nacionales y otras áreas del sector público.
    \item Analizar y desarrollar propuestas de políticas públicas basadas en evidencia y desarrollar el Informe de Competitividad 2020 y 2021. He contribuido en los capítulos de Infraestructura, Sistema de Salud, Entorno Empresarial y Logística.
 \end{itemize}


 
\end{rSection} 
\begin{rSection}{Experiencia profesional}
\textbf{Consultor} \hfill Ene 2018 - Jul 2019\\
AC Pública, subsidiaria de APOYO Consultoría \hfill \textit{Lima, Perú}
 \begin{itemize}
    \itemsep -3pt {} 
     \item Participé activamente en la elaboración y aprobación de la Política Nacional de Innovación Agraria y el Plan Nacional de Innovación Agraria.
     \item Liderar con éxito la actualización de las estimaciones de oferta y demanda de los Planes Maestros de Desarrollo de 12 aeropuertos regionales.
 \end{itemize}
\end{rSection} 

%----------------------------------------------------------------------------------------
%	TEACHING EXPERIENCE
%----------------------------------------------------------------------------------------

\begin{rSection}{Experiencia docente}

{\bf Jefe de prácticas}, Universidad del Pacífico - UP \hfill {2016 - Actualidad}\\
En los siguientes cursos y ciclos académicos:
 \begin{itemize}
    \itemsep -3pt {} 
     \item Economía General I (Ago 2016 - Dic 2018)
     \item Microeconomía I (Ago 2017 - Dic 2017)
     \item Economía General II (Ago 2019-II - Actualidad)
 \end{itemize}

\end{rSection}

%----------------------------------------------------------------------------------------
% TECHINICAL STRENGTHS	
%----------------------------------------------------------------------------------------
\begin{rSection}{Idiomas}

\begin{tabular}{ @{} >{\bfseries}l @{\hspace{6ex}} l }
Español & Nativo
\\
Inglés & Avanzado\\
\end{tabular}\\
\end{rSection}

\begin{rSection}{Paquetes informáticos y programación}

\begin{tabular}{ @{} >{\bfseries}l @{\hspace{6ex}} l }
Microsoft Excel & Advanced
\\
Eviews & Advanced\\
STATA & Advanced\\
R & Intermediate\\
Python & Intermediate\\
Atlas.ti & Intermediate\\
\end{tabular}\\
\end{rSection}

%----------------------------------------------------------------------------------------
% PROYECTOS Y PUBLICACIONES	
%----------------------------------------------------------------------------------------

\begin{rSection}{Proyectos y consultorías}
\vspace{-1.25em}
\item \textbf{Política y Plan Nacional de Innovación Agraria (Ene 2018 - Ene 2019).} {Consultoría en el marco del Programa Nacional de Innovación Agropecuaria cuyo propósito fue desarrollar un diagnóstico situacional de la innovación agropecuaria en el Perú y, con base en dicho diagnóstico, formular la Política y Plan Nacional de Innovación Agropecuaria.}
\item \textbf{Consultoría para el diseño de modelos operativos para la optimización de los procesos de gestión educativa en las Unidades de Gestión Educativa Local (Nov 2018 - Mar 2019).} {Consultoría para diseñar un nuevo modelo operativo para las Unidades de Gestión Educativa Local en Perú, con el fin de agilizar, simplificar y estandarizar cuatro procesos administrativos priorizados, que contribuyen a mejorar la calidad educativa.}
\item \textbf{Actualización de los Planes Maestros de Desarrollo de 12 aeropuertos gestionados por Aeropuertos del Perú (Dic 2018 - Jul 2019).} {Consultoría para la actualización de las estimaciones de oferta y demanda de los Planes Maestros de Desarrollo (PMD) de los 12 aeropuertos concesionados por Aeropuertos del Perú. A partir de las proyecciones de demanda de pasajeros y carga, se determinó en qué año sería necesario ampliar los aeropuertos. Para obtener más información sobre los PMD, \href{https://www.adp.com.pe/es/nuestros-proyectos}{haga click aquí.}}
\item \textbf{Consultoría para el desarrollo de estrategias que fortalezcan a la Universidad Católica Sedes Sapientiae en su filial Chulucanas (Mar 2019 - Jul 2019).} {Consultoría para desarrollar un estudio de oferta y demanda para la implementación de un nuevo campus universitario de la Universidad Católica Sedes Sapientiae (UCSS) en su sede en Chulucanas, Piura. Actualmente, la UCSS se encuentra en proceso de diseño y construcción de las nuevas instalaciones del nuevo campus en Chulucanas.}
\end{rSection} 

\pagebreak
\begin{rSection}{Proyectos y consultorías}
\item \textbf{Análisis de la situación de la niñez y adolescencia en Perú para UNICEF Perú (Jul 2020 - Feb 2021).} {Consultoría para desarrollar el análisis de la situación de la niñez y la adolescencia para el nuevo Programa de Cooperación entre el Fondo de las Naciones Unidas para la Infancia (UNICEF) y el Estado Peruano. Este análisis se centró en los derechos a la salud, la educación, la protección social y la protección contra la violencia. Puede acceder al resumen ejecutivo \href{https://www.unicef.org/peru/media/12141/file/Resumen Ejecutivo: Situaci%C3%B3n de ni%C3%B1as, ni%C3%B1os y adolescentes en el Per%C3%BA .pdf}{aquí.}}
\item \textbf{Proyecto de investigación sobre el desempeño de los funcionarios públicos a cargo de los procesos generales de contratación en el Perú (Jul 2020 - Ene 2021).} {Consultoría con el objetivo de identificar estrategias encaminadas a lograr mejoras en los procesos de contratación pública de insumos médicos. En concreto, se diseñó una intervención basada en las ciencias del comportamiento y se diseñó un plan de evaluación de impacto para medir los resultados de dicha intervención.}
\item \textbf{Propuesta de indicadores cruciales para el seguimiento de la gestión institucional del Seguro Social de Salud – EsSalud (Dic 2020 - Feb 2021).} {Consultoría para diseñar un tablero de indicadores de gestión que faciliten el seguimiento y análisis de las prestaciones en salud, para que la dirección ejecutiva de EsSalud pueda tomar mejores decisiones y mejorar la prestación de los servicios de salud.}
\item \textbf{Consultoría para la Segunda Medición del Proyecto Integración de Niños, Niñas y Adolescentes Venezolanos en Perú para UNICEF Perú (Ene 2021 - Mar 2021).} {Consultoría para evaluar la situación final del proyecto “Integración de Niñas, Niños y Adolescentes Venezolanos en Perú” implementado por UNICEF Perú. La evaluación se realizó a partir de la medición de indicadores a partir de información primaria y secundaria; así como del desarrollo de tres casos de estudio.}
\item \textbf{Consultoría para la medición de resultados del estudio de intervenciones de apoyo a procesos de compras públicas en Perú (Mar 2021 - Ene 2022).} {Consultoría para implementar y monitorear la intervención diseñada para mejorar los procesos de compras públicas de insumos médicos, así como para realizar una evaluación de impacto de la intervención. La intervención permitió un mayor índice de adjudicación de compras y una reducción en los precios unitarios de las compras de insumos médicos.}
\item \textbf{Elaboración de un análisis de impacto de la modificación del crédito a la EPS contra los aportes a EsSalud (Jun 2021 - Jul 2021).} {Consultoría para realizar un análisis de sensibilidad de los ingresos de EsSalud ante posibles cambios en el nivel de crédito que reciben las Empresas Prestadoras de Salud. Asimismo, se realizó un análisis del impacto financiero de dichos cambios y se realizaron propuestas para asegurar la sostenibilidad financiera de la institución.}
\item \textbf{Formulación del Observatorio de la iniciativa “Propuestas del Bicentenario” de Videnza Consultores para las temáticas de Estado Eficiente, Industrias Extractivas y Agricultura (Jun 2021 - Oct 2021).} {Proyecto interno de Videnza Consultores en el marco de la iniciativa “Propuestas Bicentenario”, con el propósito de generar un observatorio de indicadores que permita evaluar y dar seguimiento a 8 ejes clave con información a nivel nacional y regional. Para más información sobre el observatorio, \href{https://propuestasdelbicentenario.pe/observatorio/}{ingresa aquí.}}
\item \textbf{Consultoría para la elaboración de un documento de discusión sobre la generación de valor en el entorno de los proyectos mineros en el Perú y sus implicaciones en el conflicto (Nov 2021 - Feb 2022).} {Consultoría para identificar las condiciones y características necesarias para la generación de valor en entornos donde se desarrollan actividades mineras; así como identificar la relación que existe entre estas características y el conflicto social.}
\item \textbf{Consultoría para la elaboración de un manual de conceptos sobre relaciones sociales en el Corredor Sur Minero para MMG Las Bambas (Mar 2022 - May 2022).} {Consultoría para desarrollar un manual de conceptos que siente las bases para diseñar e implementar una estrategia de relacionamiento efectivo de la empresa MMG Las Bambas con las instituciones y actores del corredor vial sur (Apurímac - Cotabambas, Cusco - Chumbivilcas y Arequipa).}
\end{rSection} 

\pagebreak
\begin{rSection}{Proyectos y consultorías}
\item \textbf{Consultoría para la identificación del arreglo institucional para la implementación del Programa en el Corredor Minero Sur (May 2022 - Jul 2022).} {Consultoría para desarrollar un análisis de los posibles diseños u opciones institucionales para la implementación del programa de inversiones bajo un enfoque territorial, así como para realizar un análisis comparativo de los diseños u opciones institucionales identificados.}
\item \textbf{Estimación del impacto fiscal y económico de la transición energética en el Perú (Ago 2022 - Dic 2022).} {Consultoría para desarrollar la estimación del impacto fiscal y económico de la transición energética de la matriz eléctrica peruana, en términos de pérdida de recaudación de regalías gasíferas.}
\item \textbf{Consultoría para la medición de resultados del estudio RG-T1736: Intervención para apoyar los procesos de compras públicas de baja cuantía en Perú (Set 2022 - Dic 2022).} {Consultoría para implementar y monitorear la intervención diseñada para mejorar los procesos de compras públicas de menor cuantía en entidades públicas del sector salud, así como para realizar una evaluación de impacto de la intervención.}
\item \textbf{Asistencia Técnica para la integración de los procesos que conforman la Programación Multianual de Inversiones y la Programación Presupuestaria Multianual en materia de inversiones (Set 2022 - Ene 2023).} {Consultoría para elaborar una propuesta de optimización de la automatización e interrelación entre los procesos de la Programación Multianual de Inversiones y la Programación Presupuestaria Multianual en materia de inversiones}
\item \textbf{Asistencia Técnica para la exploración de las opciones de instrumentos financieros que contribuyen a la conservación y el uso sostenible de la biodiversidad, con especial atención en los insectos polinizadores (Dic 2022 - Feb 2023).} {Estudio exploratorio sobre las opciones de instrumentos financieros que contribuyen a la conservación y el uso sostenible de la biodiversidad, con especial atención en los insectos polinizadores en Perú, México, Costa Rica y Brasil.}
\item \textbf{Implementación de la tercera medición del índice de capital social 2023 a organizaciones comunitarias del ámbito DAIS (Desarrollo Alternativo Integral y Sostenible) y análisis comparativo de resultados de las tres mediciones (Mar 2023 - Actualidad).} {Consultoría para desarrollar la tercera medición del Índice de Capital Social a 24 organizaciones de productores y 34 organizaciones vecinales en el ámbito del Proyecto USAID de Fortalecimiento Institucional de DEVIDA; así como la sistematización del proceso de diseño y aplicación del Índice de Capital Social en 2019, 2021 y 2023. Este trabajo implicó la revisión documentaria y un trabajo de campo cualitativo para la identificación de fortalezas y oportunidades de mejora en la aplicación del ICS.}
\item \textbf{Consultoría para la sistematización de experiencias o prácticas de gestión pública del Proyecto de USAID: Fortalecimiento Institucional de DEVIDA (FID) (Abr 2023 - Actualidad).} {Consultoría para la sistematización de cinco experiencias de gestión pública del Proyecto FID. Las experiencias priorizadas por el Proyecto FID fueron: i) fortalecimiento de la gestión de contrataciones, ii) fortalecimiento de organizaciones de productores mediante la herramienta VEO, iii) promoción de la obtención de financiamiento en organizaciones comunitarias, iv) fortalecimiento de la gestión de inversiones en municipalidades y v) obtención de recursos adicionales municipales para la inversión pública.  Este trabajo implicó la revisión documentaria y un trabajo de campo cualitativo para la identificación de fortalezas y oportunidades de mejora en las cinco experiencias priorizadas por el Proyecto FID. }
\end{rSection} 
%----------------------------------------------------------------------------------------
\pagebreak
\begin{rSection}{Publicaciones}
Durante mi experiencia profesional he colaborado con las siguientes publicaciones:
\item Bustamante, P. (2022) \emph{La necesaria viabilidad social para el desarrollo sostenible de la minería}. En L. M. Castilla, J. Seinfield, N. Besich, C. Trivelli, J. Gallardo, A. Matsuda, . . . R. Valencia, \textit{Propuestas del Bicentenario: Rutas para el desarrollo regional} (págs. 215-241). Lima: Penguin Random House. Disponible \href{https://www.casadellibro.com/ebook-propuestas-del-bicentenario-ebook/9786125068026/13125397}{aquí.} 
\item Castilla, L. M. (2021) \emph{Reactivación Económica para el crecimiento sostenido}. En L. M. Castilla, J. Seinfield, M. von Hese, N. Besich, M. Jaramillo, R. Barrantes, . . . D. Alfaro, \textit{Propuestas del Bicentenario: Rutas para un país en desarrollo} (págs. 29-64). Lima: Penguin Random House. Disponible \href{https://www.casadellibro.com/ebook-propuestas-del-bicentenario-ebook/9786124272813/12336851}{aquí.}
\item Castilla, L. M. (2022) \emph{Encadenamientos productivos y desarrollo regional: el caso de la minería.}.  En L. M. Castilla, J. Seinfield, N. Besich, C. Trivelli, J. Gallardo, A. Matsuda, . . . R. Valencia, \textit{Propuestas del Bicentenario: Rutas para el desarrollo regional} (págs. 215-241). Lima: Penguin Random House. Disponible \href{https://www.casadellibro.com/ebook-propuestas-del-bicentenario-ebook/9786125068026/13125397}{aquí.} 
\item Castilla, L. M., Seinfeld, J., von Hese, M., Besich, N. (2021) \emph{Introducción}. En L. M. Castilla, J. Seinfield, M. von Hese, N. Besich, M. Jaramillo, R. Barrantes, . . . D. Alfaro, \textit{Propuestas del Bicentenario: Rutas para un país en desarrollo} (págs. 29-64). Lima: Penguin Random House. Disponible \href{https://www.casadellibro.com/ebook-propuestas-del-bicentenario-ebook/9786124272813/12336851}{aquí.}
\item Consejo Privado de Competitividad. (2019). \emph{Informe de Competitividad 2020}. Lima: Consejo Privado de Competitividad. Disponible \href{https://www.compite.pe/wp-content/uploads/2019/11/CPC_Peru_INC-2020_Libro-Web-Paginas.pdf}{aquí.}
\item Consejo Privado de Competitividad. (2020). \emph{Índice Regional de Gestión Pública (IRGP) 2020}. Lima: Consejo Privado de Competitividad. Disponible \href{https://www.compite.pe/wp-content/uploads/2020/07/IRGP-2020-version-final.pdf}{aquí.}
\item Consejo Privado de Competitividad. (2020). \emph{Informe de Competitividad 2021}. Lima: Consejo Privado de Competitividad. Disponible \href{https://www.compite.pe/wp-content/uploads/2021/01/Informe-de-Competitividad-2021-CPC.pdf}{aquí.}
\end{rSection} 
\begin{rSection}{Actividades extra curriculares} 
\begin{itemize}
    \item 	\textbf{EvalYouth Global Mentoring Program (EY-GMP) 2020-2021.} {El EY-GMP es una iniciativa para apoyar a los evaluadores novatos y jóvenes profesionales para que se conviertan en profesionales capacitados y confiables que puedan asumir roles de evaluación con confianza en sus comunidades y países.}
  \item 	\textbf{Impacta! Jóvenes por la gestión pública 2019-2020.} {Impacta es una asociación civil sin fines de lucro, dirigida e integrada por jóvenes convencidos de que para lograr el pleno desarrollo de un país es necesario contar con instituciones públicas sólidas, transparentes y capaces de responder con eficiencia a las necesidades de sus ciudadanos.}
  \end{itemize}


\end{rSection}


\end{document}
